
\documentclass[11pt,spanish]{article}
\usepackage[T1]{fontenc}
\usepackage{selinput}
\SelectInputMappings{%
  aacute={á},
  ntilde={ñ},
}
%%\usepackage[latin9]{inputenc}

\usepackage{array}
\usepackage{float}
\usepackage{amsmath}
\usepackage{graphicx} 

\title{Ejercicios Python Scrapping} 

\date{\vspace{-5ex}}

\begin{document}
\maketitle


\newpage
\section{Precipitaciones Vs. Incidents}
Los siguientes plots muestran la relación entre cuatro tipos de incidentes (acumulados mensualmente durante 2015) y las precipitaciones mensuales acumuladas durante el mismo año para cada uno de los condados de Maryland. La linea de ajuste corresponde a un modelo lineal simple.
\begin{table}[H]
 \centering
 \caption{Incidents Vs. Precip. Acum.}
\begin{tabular}{c}
\includegraphics[scale=1]{Plots_incident_precip.pdf}
\end{tabular}
\end{table}

\newpage
\section{Black Vs. Incidents}

Los siguientes plots muestran la relación entre cuatro tipos de incidentes (acumulados durante 2015) y la proporción de población negra para cada uno de los condados de Maryland. La linea de ajuste corresponde a un modelo lineal simple.
\begin{table}[H]
 \centering
 \caption{Incidents Vs. Precip. Acum.}
\begin{tabular}{c}
\includegraphics[scale=1]{Plot_incident_black.pdf}
\end{tabular}
\end{table}

\end{document}
